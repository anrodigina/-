\documentclass{article}     
\usepackage[utf8]{inputenc} 
\usepackage{amsfonts}
\usepackage[left=4cm,right=4cm,
    top=3cm,bottom=4cm,bindingoffset=0cm]{geometry}
\sloppy
\usepackage[T2A]{fontenc}
\usepackage{amsmath, amsfonts,amssymb,amsthm,mathtools}
\title{Домашнее задание по теории вероятностей}  
\author{Родигина Анастасия, 166 группа}     
\date{10 сентября 2017}   

\usepackage{graphicx}
\graphicspath{{pictures/}}
\DeclareGraphicsExtensions{.pdf,.png,.jpg}

\newcommand{\ip}[2]{(#1, #2)}
                             
\begin{document}            

\maketitle  
 \noindent \textbf{Задача 1}
\begin{center} 
\textit{Двадцать детей, среди которых десять мальчиков и десять девочек, садятся за круглый стол случайным образом выбирая места. Какова вероятность того, что дети сядут чередуясь: мальчик, девочка, мальчик, девочка?}
\end{center}
Для начала построим вероятностное пространство. Элементарным исходом будет $w=(w_1, w_2...)$, где $w_i$ - номер стула, на котором сидит $i-$ый ребенок. Множеством исходов будет множество перестановок длины 20 (рассаживаем детей по 20 стульям). Таким образом, $|\Omega|=20!$. Нас будут интересовать те перестановки, в которых будут чередоваться мальчики и девочки. В моей модели буду считать, что мальчики и девочки не являются одинаковыми между собой, так как речь идет о людях, а не о шариках.\\
Выберем произвольный стул (для удобства) и с него начнем рассадку детей. (По факту это равносильно тому, что мы нумеруем стулья и смотрим, что мальчики должны сидеть на четных, а девочки на нечетных номерах стульев и наоборот)\\
Найдем мощность множества интересующих нас событий. Пусть на выбранном стуле сидит девочка. Окей, тогда на нечетных стульях (относительно того магического стула) должны сидеть мальчики, а на четных - девочки. Получаем количество вариантов равное $(10!)^2$. Аналогичная ситуация будет, если мы посадим на выбранный стул девочек (так как девочка и мальчик на выбранном стуле - события независимые, количество вариантов складывается). Получаем:
$$P(A)=\frac{|A|}{|\Omega|}=\frac{2*(10!)^2}{20!}$$
\textbf{Задача 2}
\begin{center}
\textit{Из 20 ученых, среди которых 12 физиков и 8 математиков, отобрали 9 для работы в новой лаборатории. Какова вероятность того, что в лаборатории работают шесть математиков и трое физиков? Какова вероятность того, что отбор в лабораторию оказался еще более несправедливым по отношению к физикам?}
\end{center}
Определимся с вероятностным пространством. Элементарным исходом будет $w=(w_1, w_2...)$, где $w_i$ - номер ученого, который попал в лабораторию. Множеством исходов будет множество наборов длины 9 из 20 человек. Таким образом, мощность множества исходов будет равна: 
$$|\Omega|=C_{20}^9$$
Найдем количество исходов, удовлетворяющих условию: 6 математиков и 3 физика (выбираем без повторений и учета порядка группу из трех физиков и группу из 6 математиков):
$$C_{12}^3\times C_8^6$$
Таким образом,
$$P(A)=\frac{C_{12}^3\times C_8^6}{C_{20}^9}$$
Теперь найдем вероятность того, что физиков в новой лаборатории будет еще меньше (2 или 1, при этом хотя бы один физик в лаборатории точно будет, потому что математиков всего 8). Мощность множества исходов очевидно не изменится, теперь найдем количество исходов, которые удовлетворяют новому условию. Либо выбирается один физик, либо два (без повторений и учета порядка):
$$C_{12}^1\times C_8^8+C_{12}^2\times C_8^7$$
Тогда получаем вероятность:
$$P(B)=\frac{C_{12}^1\times C_8^8+C_{12}^2\times C_8^7}{C_{20}^9}$$
\textbf{Задача 3}
\begin{center}
\textit{Десять человек сели в лифт на цокольном этаже дома, в котором четыре этажа. Каждый человек на каком-то этаже выходит, но этаж выбирает случайным образом. Какова вероятность того, что на каждом этаже кто-нибудь выйдет?} 
\end{center}
Построим вероятностное пространство. Элементарным исходом будет $w=(w_1, w_2...)$, где $w_i$ - количество людей на i-м этаже. Множеством исходов будет распределение людей по четырем этажам (для каждого человека будет 4 варианта, а людей всего 10), тогда:
$$|\Omega|=4^{10}$$
 Гораздо удобнее будет посчитать вероятность того, что хотя бы на одном этаже никто не выйдет. Обозначим это событие за $\overline{A}$, а также учтем тот факт, что
 $$P(A)=1-P(\overline{A})$$
 Посчитаем количество исходов, когда ровно один этаж остается "пустым". Тогда остается распределить всех людей по остальным трем этажам. (Теперь для каждого человека остается ровно 3 варианта этажа):
  $$4\times 3^{10}$$.
  Теперь посчитаем количество исходов, когда "пустыми" остается 2 этажа (возможных комбинаций 2 свободных этажей - $C_{4}^2 = 6$, а людей остается распределить по 2 этажам):
  $$6\times2^{10}$$
  Теперь рассмотрим ситуацию, когда люди все выйдут на одном этаже, таких вариантов ровно 4.
  Воспользуемся формулой включений-исключений:
  $$|\overline{A}| = 4\times 3^{10} - 6\times2^{10} + 4$$
  Тогда $$P(A) = 1 - \frac{4\times 3^{10} - 6\times2^{10} + 4}{4^{10}}$$
\end{document}







